\appendix

\section{Appendices}
\label{sec:appendices}

This appendix provides reproducibility instructions and documents the mechanized anchors referenced in the main text.

\subsection{Reproducibility: regenerating artifacts and paper tables}
\label{sec:appendix-repro}

All experiments write artifacts under \texttt{artifacts/} (JSON and CSV). The recommended end-to-end regeneration sequence from the repository root is:

\begin{quote}
\texttt{python python/scripts/run\_all\_exhibits\_smoke.py}
\end{quote}

This runner executes the exhibit scripts and verifies that the expected artifacts exist, including at minimum:

\begin{itemize}
  \item \texttt{artifacts/exhibit\_dpi\_smoke/metadata.json}
  \item \texttt{artifacts/exhibit\_clock\_budget\_smoke/metadata.json}
  \item \texttt{artifacts/exhibit\_enablement\_birth\_smoke/metadata.json}
  \item \texttt{artifacts/exhibit\_constraints\_cones\_smoke/metadata.json}
  \item \texttt{artifacts/exhibit\_no\_global\_time\_smoke/metadata.json}
  \item \texttt{artifacts/exhibit\_no\_signalling\_toy/metadata.json}
  \item \texttt{artifacts/sweeps/sweep\_smoke/results.csv}
  \item \texttt{artifacts/sweeps/sweep\_smoke/summary.json}
\end{itemize}

Paper tables are generated from these artifacts by:

\begin{quote}
\texttt{python python/scripts/paper/make\_paper\_tables.py}
\end{quote}

The generated tables are written to \texttt{docs/paper/tables/} and included in the manuscript via \texttt{\textbackslash input\{tables/...\}}.

\subsection{Mechanized anchors (Lean)}
\label{sec:appendix-mechanized}

We include lightweight Lean~4 formalizations as structural anchors for several claims.
These files are intended to mechanize small algebraic skeletons cited in the narrative, not to verify the empirical audits end-to-end.

\paragraph{Holonomy obstruction (no global time).}
\begingroup\sloppy
File: \url{lean/TimeWorld/HolonomyNoGlobalTime.lean}. Key identifiers:
\url{triangle_sum_of_potential} and \url{no_global_potential_of_nonzero_triangle_holonomy}.
\endgroup
These capture the telescoping identity for exact 1-forms and the obstruction implied by nonzero cycle sum.

\paragraph{Closure descent to fixed points.}
\begingroup\sloppy
File: \url{lean/TimeWorld/DescentToFixpoints.lean}. Key identifiers:
\url{map_fix_of_commute} and \url{restrictToFix}.
\endgroup
These encode a basic ``descent'' fact: if an idempotent packaging map commutes with an update, then the update restricts to the packaged fixed-point subspace.

\paragraph{Ledger preorder anchor.}
\begingroup\sloppy
File: \url{lean/TimeWorld/LedgerPreorder.lean}. Key identifiers:
\url{ledgerPreorder} and \url{ledger_step_le_of_monotone}.
\endgroup
These show how a monotone ledger induces a preorder compatible with an update rule.

\paragraph{No-signalling toy anchors.}
\begingroup\sloppy
File: \url{lean/TimeWorld/NoSignallingToy.lean}. Key identifiers:
\url{marginalB_uniform_of_xor_constraint} and \url{signalling_marginalB_depends_on_x}.
\endgroup
These formalize, in a minimal Boolean setting, that constraint-mediated sharp conditionals do not imply a signalling channel.

\subsection{Code map (Python)}
\label{sec:appendix-code}

The main Python components are organized under \texttt{python/src/time\_world/}:

\begin{itemize}
  \item \texttt{model.py}: toy Markov world construction and simulation
  \item \texttt{audits\_ep.py}: stationary distribution and entropy production
  \item \texttt{audits\_path\_kl.py}: DPI-safe path-reversal KL estimation under lenses
  \item \texttt{clock\_audits.py}: clock viability metrics, including progress/anti-stall rates
  \item \texttt{enablement.py}: closure defect and forced theory extension
  \item \texttt{constraints\_cones.py}: constraint masks and reachability cones
  \item \texttt{holonomy.py}: protocol holonomy measurement
  \item \texttt{no\_signalling\_toy.py}: constraint vs signalling boxes
\end{itemize}

Exhibit scripts live under \texttt{python/scripts/}. See \texttt{docs/experiments/index.md} for the internal runbook list.
